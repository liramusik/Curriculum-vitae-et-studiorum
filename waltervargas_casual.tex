%% start of file `jdoe_casual.tex'.
%% Copyright 2006 Xavier Danaux.
%
% This work may be distributed and/or modified under the
% conditions of the LaTeX Project Public License version 1.3c,
% available at http://www.latex-project.org/lppl/.


\documentclass[11pt]{moderncv}

% moderncv styles
\moderncvstyle{casual}         % optional argument are 'nocolor' (black & white cv) and 'roman' (for roman fonts, instead of sans serif fonts)
%\moderncvstyle{classic}       % idem

% character encoding
\usepackage[utf8]{inputenc}   % replace by the encoding you are using

% personal data (the given example is exhaustive; just give what you want)
\firstname{Walter}
\familyname{Vargas}
\title{Técnico Superior en Informática\dots}
%\address{12 somestreet\\3456 somecity} % for classic style
\address{Venezuela}  % for casual style
\phone{+58(416)5023755}
\email{walter@covetel.com.ve}
\extrainfo{\weblink{blogs.covetel.com.ve/walter}}
\photo[64pt]{jdoe_picture} % also optional, and the optional argument is the height the picture must be resized to
\quote{Alguna frase ...}% also optional

%\renewcommand{\listsymbol}{{\fontencoding{U}\fontfamily{ding}\selectfont\tiny\symbol{'102}}} % define another symbol to be used in front of the list items

% the ConTeXt symbol
\def\ConTeXt{%
  C%
  \kern-.0333emo%
  \kern-.0333emn%
  \kern-.0667em\TeX%
  \kern-.0333emt}

% slanted small caps (only with roman family; the sans serif font doesn't exists :-()
%\usepackage{slantsc}
%\DeclareFontFamily{T1}{myfont}{}
%\DeclareFontShape{T1}{myfont}{m}{scsl}{ <-> cork-lmssqbo8}{}
%\usefont{T1}{myfont}{m}{scsl}Testing the font

% command and color used in this document, independently from moderncv 
\definecolor{see}{rgb}{0.5,0.5,0.5}% for web links
\newcommand{\up}[1]{\ensuremath{^\textsf{\scriptsize#1}}}% for text subscripts

%----------------------------------------------------------------------------------
%            content
%----------------------------------------------------------------------------------
\begin{document}
\maketitle
\makequote

\section{Información Personal}
\cvitem{Lugar de Nacimiento}{\small Caracas - República Bolivariana de Venezuela }
\cvitem{Fecha de Nacimiento}{\small 17 de Febrero de 1985 }
\cvitem{Estado Civil}{\small Soltero}
\cvitem{Cédula de Identidad}{\small V-16.612.574}
\cvitem{GNU/Linux User ID}{\small 268566}
\cvitem{Dirección}{\small Estado Táchira, Urb. Altos de Paramillo, manzana 18, parcela 24, calle 7 los sauces.}
\cvitem{Correo Electrónico}{\small waltervargas@gmail.com}
\cvitem{Skype}{\small waltervargasm}

\section{Educación}
\cventry{Actualidad}{Ingeniería en Informática}{Instituto Universitario de Tecnología Agro Industrial Región los Andes}{}{}{ Esperando obtener el título para finales del 2011}
\cventry{2002--2009}{Técnico Superior en Informática}{Instituto Universitario de Tecnología Agro Industrial Región los Andes}{}{}{}

\section{Lenguajes}
\cvlanguage{Español}{Nativo}{Lectura, Escritura, Comprensión}
\cvlanguage{Ingles}{Intermedio}{Lectura, Escritura, Comprensión}
\closesection{}
\pagebreak{}

\section{Habilidades en Computación}
\cvcomputer{\bfseries{Sistemas Operativos}}{
	Debian GNU/Linux (Especialmente), 
	RedHat, 
	Mandriva, 
	Unix System V, 
	Windows NT-2000-XP,
	BeOS,
	Linux Router Project,
	HP-UX,
	Cisco OS
}{\bfseries{Administración}}{
	DNS, DHCP,
	Open LDAP, 
	Sendmail, 
	Postfix, 
	Dovecot, 
	Samba, 
	OpenVPN, 
	OpenSSL, 
	OpenCA, 
	Git, 
	Subversion, 
	Trac, 
	Apache (Mod SSL, Mod Perl, Mod PHP), 
	Mailman, 
	Request Tracker, 
	Squid, 
	SquidGuard, 
	Iptables, 
	Ipchains, 
	NMAP,
	WebGUI
}
\cvcomputer{\bfseries{Programación}}{
	\textsc{Pascal}, 
	C,
	C++, 
	SQL
}{\bfseries{Scripting}}{
	JavaScript, 
	Perl, 
	PHP, 
	Shell, 
 	Ruby
}
\cvcomputer{\bfseries{Librerías}}{
	LWP, LWP::Simple
	Net::LDAP,
	DBI,
	DBIx::Class,
	Catalyst,
	HTML::FormFu,
	Config::General,
	WWW::Mechanize,
	XML::LibXML,
	Catalyst::Controller::REST,
	Catalyst::View::GD,
	Catalyst::Model::LDAP,
	Image::Resize,
	GD,
	Moose,
	QT,
	GTK+,
	Statistics::Zscore,
	Data::Dumper,
	Spreadsheet::Read,
}{\bfseries{Tipografía y Diseño}}{
	\LaTeX, 
	Scribus, 
	Inkscape, 
	Gimp
}
\cvcomputer{\bfseries{Desarrollo Web}}{
	Catalyst Framework,
	XML, 
	XLST, 
	XPATH, 
	HTML, 
	XHTML, 
	CSS, 
	AJAX, 
	Patrones de Diseño
}{\bfseries{Base de Datos}}{
	MySQL, 
	PostgreSQL, 
	SQLite, 
	Berkeley DB, 
	CouchDB
}

\section{Habilidades en Electrónica}
\cvcomputer{\bfseries{Instrumentación}}{
	Manejo de Cautin, Multimetro y construcción básica de circuitos impresos.
}{\bfseries{Radio Difusión}}{
	Operación de equipos de radio difusión como consolas mezcladoras, ecualizadores, filtros, y equipos de trasmisión como FM/AEQ, micrófonos, construcción de cables XLR a medida.  
}
\cvcomputer{\bfseries{Micro Controladores}}{
	Programación de microcontroladores PIC16F84, PIC16F877, ARM.
}{\bfseries{Microondas}}{
	Fundamentos básicos de microondas, transmisión y propagación de ondas.
}
\cvcomputer{\bfseries{Antenas}}{
	Construcción de antenas del tipo: Yagi, Omni, Guia de ondas sectorial y Biquad para las frecuencias 2.4 y 5.8 Ghz.
}{\bfseries{Sistemas Satelitales}}{
	Instalación, operación y mantenimiento de las estaciones remotas del \bfseries{Satélite Simón Bolivar}
}

\section{Experiencia}

\cventry{Diciembre 2007--\\Actualidad}{Coordinador General}{\bfseries{Asociación Cooperativa Venezolana de Tecnologías Libres R.S. (COVETEL)}}{San Cristóbal}{ Responsable de las siguientes áreas: Coordinación, Gerencia, Ventas, Proyectos }{}{}

\cventry{Julio 2010--\\Actualidad}{Contratista}{\bfseries{Centro Nacional de Tecnologías de Información (CNTI)}}{San Cristóbal}{Soporte de 4to Nivel bajo la figura de acompañamiento para la Gerencia de Plataforma Tecnológica CNTI}{Tecnologías Involucradas: XEN, KVM, PROXMOX, OCS Inventory, PostgreSQL, Mysql, Perl, PHP, Asesoría para la creación de planes de prevención y contingencia de los servicios, Mantener la disponibilidad de los servicios de las plataformas, Correo, DHCP, DNS, Firewalls, OpenLDAP, Puppet, CFEngine, Trac, SVN, GIT, MoinMoin, Request Tracker, OpenLDAP + Samba, VPN, Firewall, Infraestructura de PKI, Aseguramiento de aplicaciones, Infraestructura de Autenticación y Autorización LDAP, Pruebas de Penetración, Assesment de Vulnerabilidades, Hardening de Servicios, SGSI + ISO 27001-2, Investigación Forense, Telefonía IP, Nagios, Zabbix, OSSIM, Heartbeat V2, DRBD, Linux-HA, OCFS2, iSCSI, Cluster XEN, Open Cluster Framework.}{}

\cventry{Junio 2010--\\Actualidad}{Contratista}{\bfseries{Centro Nacional de Tecnologías de Información (CNTI)}}{San Cristóbal}{Desarrollo del Sistema para la Elaboración Automatizada de Carnets Institucionales basado en LDAP}{Tecnologías Involucradas: OpenLDAP, Perl, Catalyst Framework, HTML::FormFu, Moose, REST, GD, File::Tar, XHTML, CSS, AJAX}{}

\cventry{Enero 2010--\\Actualidad}{Contratista}{\bfseries{Centro Nacional de Tecnologías de Información (CNTI)}}{San Cristóbal}{Desarrollo del Sistema para la Automatización del protocolo de prueba de portales de Internet | Nota: Este proyecto lo desarrolla el equipo de Covetel al cual pertenezco}{Tecnologías Involucradas: Perl, Catalyst Framework, HTML::FormFu, Moose, PostgreSQL, DBIx::Class, REST, SOAP, WSDL, OpenLDAP}{}

\cventry{Enero 2010--\\Actualidad}{Contratista}{\bfseries{Compañía Anónima Nacional Teléfonos de Venezuela (CANTV)}}{San Cristóbal}{Despliegue Satelital Simón Bolívar}{Responsabilidades: Instalación, Puesta a Punto y Mantenimiento de las terminales terrestres del Satélite Simón Bolivar en el estado Táchira}{}

\cventry{Noviembre 2009--\\Diciembre 2009}{Contratista}{\bfseries{Compañía Anónima Nacional Teléfonos de Venezuela (CANTV)}}{Caracas}{Fue Dictado un Curso Especializado e Intensivo de Xen - LDAP - Correo - Samba - Alta Disponibilidad al Equipo del Proyecto de Migración a Software Libre CANTV}{Tecnologías Involucradas: XEN, OpenLDAP, Postfix, Dovecot, Samba}{}

\cventry{Mayo 2009--\\Agosto 2009}{Contratista}{\bfseries{Compañía Anónima Nacional Teléfonos de Venezuela (CANTV) | Presidencia de la República Bolivariana de Venezuela}}{Caracas}{Asesorías para la creación de una plataforma de Telefonía IP basada en Asterisk con SER, escalable para la Presidencia de la República Bolivariana de Venezuela}{Tecnologías Involucradas: XEN, OpenLDAP, Postfix, Dovecot, Samba}{}

\cventry{Noviembre 2008--\\Enero 2010}{Contratista}{\bfseries{Ministerio del Poder Popular para Economía y Finanzas}}{Caracas}{Migración y Acondicionamiento de la Plataforma de Servicios}{Actividades (Realizadas por el equipo de Covetel, al cual pertenezco): Implementación de un Laboratorio de Migración, Adecuación de la Plataforma de Virtualización, Migración de cuentas de usuario desde AD Windows 2000 a OpenLDAP, Migración de los buzones al nuevo servidor de correo, Implementación de un Proxy con control parental, Implementación de un servidor de correo Postfix + Dovecot, Implementación servidor DNS Primario utilizando Bind integrado a OpenLDAP, Implementación de un servidor DNS Cache, Implementación de un servidor SAMBA PDC, Transferencia Tecnológica y Soporte a la plataforma durante ese periodo
 }{}

\cventry{Junio 2009--\\Diciembre 2009}{Contratista}{\bfseries{Centro Nacional de Tecnologías de Información (CNTI)}}{Caracas}{Construcción del Kit de Servicios, Nota: Este proyecto fue realizado por el equipo de Covetel, al cual pertenezco}{Partes del Kit de Servicios: Xen, OpenLDAP, TinyCA, Postfix, Dovecot, Thunderbird, RoundCube, Spamassassin, Dspam, ClamAV, Amavis-new, NTP, Firewall (Iptables), OpenVPN, DNS Integrado a LDAP, DHCP Integrado a LDAP, FreeRadius y autenticación 802.1X, Apache2 (Autenticación y VHOST integrados a LDAP), Samba (PDC, BDC, Servidor de Impresion y Servidor de Archivos), FAI, Puppet, SystemImager}{}

\cventry{Julio 2009--\\Agosto 2009}{Apoyo Voluntario}{\bfseries{Superintendencia de Servicios de Certificación Electrónica}}{Caracas}{Asesoría de seguridad durante la implementación de los servicios web del proyecto para consejos comunales}{Tecnologías Involucradas: Firewall, MySQL, PHP5, Hardening, Balanceo de Carga con DNS usando multiples punteros de tipo A }{}

\cventry{Noviembre 2006--\\Diciembre 2007}{Asesor Externo}{\bfseries{Distrito Tecnológico Social AIT PDVSA Mérida}}{Mérida}{Implementación de la Infraestructura de Networking y asesor en las siguientes áreas: Seguridad Lógica, Servicios de Red, Redes Inalámbricas}{}{}

\cventry{Noviembre 2004--\\Julio 2006}{Desarrollador}{\bfseries{IntranetHome C.A. - Contratista DirectTV}}{San Cristóbal}{Desarrollador Líder, de la aplicación de control de instalaciones y servicios para las contratistas de DIRECTV de la región Occidental}{Tecnologías Involucradas: PHP5, JAVA SCRIPT, PERL, XHTML, CSS, SQL.}{}

\cventry{Durante el año 2006}{Facilitator}{\bfseries{Universidad Valle del Momboy}}{Valera - Trujillo}{Diplomado de Software Libre}{}{}

\cventry{Noviembre 2005--Diciembre 2005}{Administrador de Red}{\bfseries{Juegos Nacionales Andes 2005}}{San Cristóbal}{Implementación y administración de los servidores que controlaban la conectividad de las infraestructuras deportivas}{}{}

\cventry{Durante el año 2004}{Miembro Fundador}{\bfseries{Cooperativa de Investigación y Desarrollo de Telecomunicaciones CIDTEL 546}}{San Cristóbal}{}{}{}

\cventry{2000--2002}{Administrador de Red}{AirNet C.A.}{San Cristóbal}{Responsabilidades: Administrar una red inalámbrica implementada usando el estandar 802.11b, utilizando Linux Router Proyect como terminales remotos para los clientes}{Tecnologías Involucradas: Ipchains, Iptables, LRP, Kernel Linux 2.2.9, Squid, DNS}{}


\section{Eventos}

\cventry{19 al 23 de Noviembre del 2007}{Ponente}{V Foro Mundial de Tecnologías Libres}{Puerto Ordaz / Centro de Convensiones Hotel Intercontinental Guayana.}{Tema: Proyecto Redes Libres del Sur}{}{}

\cventry{20 y 21 de Julio del 2007}{Ponente}{3er Congreso Nacional de Software Libre (CNSL 3)}{Zulia / Museo de Arte del Zulia}{Tema: Redes Inalámbricas con GNU/Linux}{}{}

\cventry{15 y 16 de Junio del 2007}{Ponente}{3er Congreso Nacional de Software Libre (CNSL 3)}{Mérida / ULA}{Tema: Redes Inalámbricas con GNU/Linux}{}{}
\cventry{08 y 09 de Junio del 2007}{Ponente}{3er Congreso Nacional de Software Libre (CNSL 3)}{Barinas / UNELLEZ}{Tema: Redes Inalámbricas con GNU/Linux}{}{}\cventry{08 y 09 de Junio del 2007}{Ponente}{3er Congreso Nacional de Software Libre (CNSL 3)}{Barinas / UNELLEZ}{Tema: Redes Inalámbricas con GNU/Linux}{}{}
\cventry{26 al 28 de Mayo del 2006}{Ponente}{2do Congreso Nacional de Software Libre (CNSL 2)}{Barquisimeto / Lara}{Tema: Redes Inalámbricas con GNU/Linux}{}{}
\cventry{19 al 21 de Mayo del 2006}{Ponente}{2do Congreso Nacional de Software Libre (CNSL 2)}{Valera / Trujillo}{Tema: Redes Inalámbricas con GNU/Linux}{}{}
\cventry{12 al 14 de Mayo del 2006}{Ponente}{2do Congreso Nacional de Software Libre (CNSL 2)}{Mérida}{Tema: Redes Inalámbricas con GNU/Linux}{}{}
\cventry{5 al 7 de Mayo del 2006}{Ponente}{2do Congreso Nacional de Software Libre (CNSL 2)}{Valera / Trujillo}{Tema: Redes Inalámbricas con GNU/Linux}{}{}
\cventry{28 al 30 de Abril del 2006}{Ponente}{2do Congreso Nacional de Software Libre (CNSL 2)}{San Cristóbal / Táchira}{Tema: Redes Inalámbricas con GNU/Linux}{}{}
\cventry{18 al 20 de Enero}{Ponente}{VI Congreso de Expotecnología UVM}{Valera / Trujillo}{Tema: Redes Inalámbricas con GNU/Linux}{}{}
\cventry{30 de Junio y 1 de Julio 2005}{Ponente}{1er Congreso Nacional de Software Libre}{Mérida}{Tema: Redes Inalámbricas con GNU/Linux}{}{}
\cventry{20 y 21 de Junio 2005}{Ponente}{1er Congreso Nacional de Software Libre}{Barinas}{Tema: Redes Inalámbricas con GNU/Linux}{}{}
\cventry{15 de Marzo 2005}{Ponente}{III Congreso de Gerencia Informática}{San Cristóbal}{Tema: Redes Inalámbricas con GNU/Linux}{}{}
\cventry{26 de Febrero 2005}{Ponente}{2do Foro Regional de Tecnología Libre}{San Cristóbal}{Tema: Redes Inalámbricas con GNU/Linux}{}{}
\cventry{26 de Febrero 2005}{Organizador}{2do Foro Regional de Tecnología Libre}{San Cristóbal}{Tema: Redes Inalámbricas con GNU/Linux}{}{}
\cventry{12 de Noviembre 2004}{Organizador}{1er Foro Regional de Tecnología Libre}{San Cristóbal}{Tema: Redes Inalámbricas con GNU/Linux}{}{}


\section{Intereses}
\cvitem{Diseño}{\small Amante del Diseño y la Fotografía.}
\cvitem{Deportes}{\small Natación, Montañismo, Parapente, Paracaidismo, Motocross}
\cvitem{Bricolaje}{\small Amante de las maquinas y herramientas para trabajo con madera, metal y plastico}

\nocite{*}
\bibliographystyle{plain}
\bibliography{jdoe_publications}

\end{document}

%% end of file `jdoe_casual.tex'.
